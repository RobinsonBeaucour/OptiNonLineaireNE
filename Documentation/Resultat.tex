\documentclass{article}
\usepackage[utf8]{inputenc}
\usepackage{amsmath}
\usepackage{amsfonts}
\usepackage{geometry}
\usepackage{stmaryrd}
\usepackage[table]{xcolor}
\usepackage[french]{babel}
\usepackage{fancybox}
\usepackage{tikz}
\usepackage{listings}
\usepackage{adjustbox}
\usepackage{amsmath}
\usepackage{multicol}
\usepackage{multicol-floats}
\usepackage{lipsum}
\usetikzlibrary{positioning}
\geometry{a4paper,left=25mm,right=25mm,top=20mm}

\title{Planification du pompage dans un réseau de distribution d'eau potable ramifié\\-\\Optimisation non-linéaire en nombre entier}
\author{Robinson Beaucour}
\date{Décembre 2022}

\begin{document}

\maketitle

\vspace{2cm}

\begin{abstract}
    Ce document fournit les méthodes et les résultats du projet d'optimisation non-linéaire réalisé dans le cadre du Mastère Spécialisé en Optimisation des Systèmes Energétiques de Mines ParisTech.
    Ce document fournit une decription du problème d'optimisation traité, discute des relaxations réalisées pour rendre le problème plus facilement résoluble. L'algorithme solveur et les solutions qu'il renvoit sont ensuite discuté.
\end{abstract}


\clearpage  

\begin{center}
    \large \textbf{4WT Problem}
\end{center}
\begin{figure}[h]
    \centering
    \begin{tikzpicture}[main/.style = {draw, circle}] 
        \node[main] (s) at (0,0) {$s$};
        \node[main] (j_1)   at (2,1)    {$j_1$}; 
        \node[main] (j_2)   at (5,1)    {$j_2$};
        \node[main] (r_1)   at (1.5,3)  {$r_1$};
        \node[main] (r_4)   at (2.5,2)  {$r_4$};
        \node[main] (r_2)   at (4.5,3)  {$r_2$};
        \node[main] (r_3)   at (5.5,2)  {$r_3$};
        \draw   [line width=0.5mm]  (s)     --  (j_1);
        \draw   [line width=0.5mm]  (j_1)   --  (j_2);
        \draw   [line width=0.5mm]  (j_1)   --  (r_1);
        \draw   [line width=0.5mm]  (j_1)   --  (r_4);
        \draw   [line width=0.5mm]  (j_2)   --  (r_2);
        \draw   [line width=0.5mm]  (j_2)   --  (r_3);
    \end{tikzpicture}
    \caption{Réseau de distribution simple (4WT)}
\end{figure}

\textbf{Variables de décision}
$$
\left.
    \begin{array}{lll}
        Q_{pompe,t}^{(k)}       &   \text{Débit sortant de la pompe $k$ à l'instant $t$}    & [Q_{min}^{(k)},Q_{max}^{(k)}]\cup\{0\}\\[0.2cm]
        Q_{reserv,t}^{(r)}      &   \text{Débit entrant du réservoire $r$ à l'instant $t$}  & \mathbb{R}_+\\[0.2cm]
        Q_{jonction,t}^{(n,n')}   &   \text{Débit dans le tuyau allant du noeud $n$ au $n'$ à l'instant $t$}  &   \mathbb{R}_+\\[0.2cm]
        C_t^{(n)}               &   \text{Niveau de charge (en m) au noeud $n$}             & \mathbb{R}_+\\[0.2cm]
        G_{pompe,t}^{(k)}       &   \text{Gain de charge de la pompe $k$ à l'instant $t$}   & \mathbb{R}_+\\[0.2cm]
        P_{pompe,t}^{(k)}       &   \text{Puissance électrique consommée par la pompe $k$ à l'instant $t$}& \mathbb{R}_+\\[0.2cm]
        V_t^{(r)}               &   \text{Volume du réservoire $r$ à l'instant $t$}& [V_{min}^{(r)},V_{max}^{(r)}]\\[0.2cm]
        S_{on,t}^{(k)}          &   \text{Etat de la pompe $k$ (allumé/éteint) à l'instant $t$}&\{0,1\}\\[0.2cm]
    \end{array}
\right.
$$
\textbf{Contraintes}
\begin{equation}
    \tag{Equilibre flux}  
    \left.
        \begin{array}{lcccc}
            \forall t,\forall j   &   \sum_{n} Q_{jonction,t}^{(n,j)}    & = &   \sum_{n} Q_{jonction,t}^{(j,n)}\\[0.2cm]
        \end{array}
    \right.
\end{equation}

\begin{equation}
    \tag{Satisfaction demande}  
    \left.
        \begin{array}{lcccc}
            \forall t, \forall r   &   V_{t+1}^{(r)}-V_t^{(r)}     & = &   Q_{reserv,t}^{(r)} - D_t^{(r)}\\[0.2cm]
        \end{array}
    \right.
\end{equation}
    
\begin{equation}
    \tag{Conso. élec pompe}  
    \left.
        \begin{array}{lccc}
            \forall t, \forall k   &   P_{pompe,t}^{(k)}     & = &   \Gamma_0^{k}S_{on,t}^{(k)} + \Gamma_1^{(k)}Q_{pompe,t}^{(k)}\\[0.2cm]
        \end{array}
    \right.
\end{equation}

\begin{equation}
    \tag{Gain charge pompe}
    \left.
        \begin{array}{lccc}
            \forall t, \forall k   &   G_{pompe,t}^{(k)}     & = &   \psi_0^{k}S_{on,t}^{(k)} + \psi_2^{(k)}(Q_{pompe,t}^{(k)})^2\\[0.2cm]
        \end{array}
    \right.
\end{equation}

\begin{equation}
    \tag{Perte charge flux}
    \left.
        \begin{array}{lcccc}
            \forall t, \forall n,n'   &   C_t^{(n)}  - C_t^{(n')}    & = &   \phi_1^{(n,n')}Q_{jonction,t}^{(n,n')} + \phi_2^{(n,n')}(Q_{jonction,t}^{(n,n')})^2\\[0.2cm]
        \end{array}
    \right.
\end{equation}

\begin{equation}
    \tag{Charge source}
    \left.
        \begin{array}{lcccc}
            \forall t, \forall n,n'   &   0 & = & (\sum_k S_{on,t}^{(k)})    (G_{pompe,t}^{(k)}-C_t^{(s)})\\[0.2cm]
        \end{array}
    \right.
\end{equation}

\begin{equation}
    \tag{Charge jonction}
    \left.
        \begin{array}{lcccc}
            \forall t, \forall j   &   C_t^{(j)}    & \geq &   H^{(j)}\\[0.2cm]
        \end{array}
    \right.
\end{equation}

\begin{equation}
    \tag{Charge réservoire}
    \left.
        \begin{array}{lcccc}
            \forall t, \forall r   &   C_t^{(r)}    & \geq &   H^{(j)}\\[0.2cm]
        \end{array}
    \right.
\end{equation}
    
% \underline{GAMS :}\\
\begin{adjustbox}{max width=\textwidth}
    \begin{lstlisting}
        Noeud(j,t) .. sum(n$l(j,n), Qpipe(j,n,t)) =e=  sum(n$l(n,j), Qpipe(n,j,t));
        Satisfaction_demande(r,t) ..  v(r,t) - v(r,t-1) - vinit(r,t) =e= 1 * (sum=(n\$l(n,r),Qpipe(n,r,t))-demand(r,t));
        Elec_pompe(k(c,d),t) .. Ppompe(k,t) =g=  gamma(c,"0") * Son(k,t) + gamma(c,"1")*Qpompe(k,t);
        Gain_charge_pompe(k(c,d),t) .. Gpompe(k,t) =l= psi(c,"0") * Son(k,t) + psi(c,"2")*Qpompe(k,t)**2;
        Charge_s("s",t) .. 0 =l= (sum(k, Gpompe(k,t))-Charge("s",t))*sum(k, Son(k,t));
    \end{lstlisting}
\end{adjustbox}
\clearpage
\textbf{Objectif}
$$
    \text{Minimiser }   \sum_t \sum_k P_{pompe,t}^{(k)}\cdot C_t
$$

\textbf{Amélioration de la résolvance du problème}
\begin{multicols}{2}
    Le problème est initialement non-linéaire en nombre entier et non convexe. Ce qui rend les solutions difficiles à trouver. Pour améliorer la résolvance du problème, il est possible de limiter le nombre de combinaisons de nombre entier possible par l'équation :
    \begin{equation}
        \tag{Symétrie}
        \forall k, \forall t, S_{on,t}^{(k+1)}\leq S_{on,t}^{(k)}
    \end{equation}
    Plusieurs relaxations convexes permettent de rendre le problème convexe, ce qui permet de considérablement réduire le temps pour trouver une solution optimale.\\
    Par ailleurs, l'équation définissant la charge au niveau de la source peut être linéarisée par l'introduction d'un majorant $M$ déterminé par le charge maximale que les pompes peuvent fournir.
\end{multicols}
Relaxation convexe:
\begin{equation}
    \tag{Gain charge pompe relax.}
    \left.
        \begin{array}{lccc}
            \forall t, \forall k   &   G_{pompe,t}^{(k)}     & \leq &   \psi_0^{k}S_{on,t}^{(k)} + \psi_2^{(k)}(Q_{pompe,t}^{(k)})^2\\[0.2cm]
        \end{array}
    \right.
\end{equation}
\begin{equation}
    \tag{Perte charge flux relax.}
    \left.
        \begin{array}{lcccc}
            \forall t, \forall n,n'   &   C_t^{(n)}  - C_t^{(n')}    & \leq &   \phi_1^{(n,n')}Q_{jonction,t}^{(n,n')} + \phi_2^{(n,n')}(Q_{jonction,t}^{(n,n')})^2\\[0.2cm]
        \end{array}
    \right.
\end{equation}
\begin{equation}
    \tag{Charge source relax.}
    \left.
        \begin{array}{lcccc}
            \forall t, \forall n,n'   &   C_t^{(s)} & = & G_{pompe,t}^{(k)} + M S_{on,t}^{(k)}\\[0.2cm]
        \end{array}
    \right.
\end{equation}
\textbf{Méthode de résolution}\\
\begin{multicols}{2}
    Le problème est retranscrit dans un fichier GAMS. La solution au problème est acquise en utilisant le solver BARON sur les serveurs NEOS (https://neos-server.org/neos/solvers/minco:BARON/GAMS.html).\\
    Le serveur NEOS propose BARON pour la solution des problèmes d'optimisation à contraintes non linéaires d'entiers mixtes et des problèmes d'optimisation globale. Les problèmes pour BARON peuvent être soumis sur NEOS au format AMPL ou GAMS.\\
    BARON est un système de calcul pour résoudre des problèmes d'optimisation non convexes à l'optimalité globale. Les problèmes non linéaires purement continus, purement entiers et mixtes peuvent être résolus avec le logiciel. Le navigateur d'optimisation de branche et de réduction tire son nom de la combinaison de la propagation des contraintes, de l'analyse d'intervalle et de la dualité dans son arsenal de réduction avec des concepts de branche et de liaison améliorés alors qu'il serpente à travers les collines et les vallées de problèmes d'optimisation complexes à la recherche de solutions globales.\\
    Le solver renvoie une solution réalisable dont l'écart
    du coût avec le coût de la solution optimale est majoré. Cette majoration est garantie par la majoration de l'écart entre la solution réalisable et la solution optimale pour le problème relaxé (nombre entier relaxé).
\end{multicols}

\textbf{Résultats}\\

\begin{multicols}{2}
\lipsum[1]
\multicolfloat{1}{2}{\centering
  \includegraphics{images/4WT_tol05_state.PNG}
%  \captionof{figure}{A test}
}
\end{multicols}
\lipsum[1]
\textbf{Discussion}\\
\end{document}


